\documentclass[a4paper,10]{article}
\usepackage{amsmath}
\usepackage{graphicx}
\usepackage[utf8x]{inputenc}
\usepackage[T1]{fontenc}
\usepackage{multicol}
\usepackage{environ}
\usepackage{latexsym,amsmath,amssymb}
\usepackage{fp,numprint}
\FPmessagesfalse  % pour ne pas voir les messages de FP
\usepackage{tikz}
%\usepackage[francais,bloc,completemulti]{automultiplechoice} 
% completemulti : la réponse qui se met automatiquement "aucune de ces réponses n'est correcte"
\usepackage[francais,bloc]{automultiplechoice}
\usepackage{multicol}
%\setmainfont{Linux Libertine O}
\geometry{hmargin=1.5cm,headheight=2cm,headsep=.3cm,footskip=1cm,top=2.5cm,bottom=2.5cm}
% les plus 
\usepackage{tabto}
\usepackage{tikz}
%%%%%%%%%% Start TeXmacs macros
\newcommand{\tmop}[1]{\ensuremath{\operatorname{#1}}}
\newcommand{\tmtextbf}[1]{{\bfseries{#1}}}
\newcommand{\tmtextup}[1]{{\upshape{#1}}}
%%%%%%%%%% End TeXmacs macros
%%%
\begin{document}
\AMCrandomseed{1527384}
%\AMCtext{none}{None of the above.}
\def\AMCbeginQuestion#1#2{\par \hspace*{0.2cm}{\bf #1:} #2}
%\def\AMCbeginAnswer#1#2{\tabto{13cm}\hspace{0mm}#2}
\def\AMCbeginAnswer#1#2{\tabto{13cm}\tabto{13cm}#2}
\AMCformHSpace=.5em
\AMCinterBrep=5.5ex
\def\AMCanswer#1#2{#1\hspace{2mm}#2\hspace{1cm}}
%%%%

%%%
%\def\AMCformAnswer#1{\hspace{\AMCformHSpace} #1}
%
%\AMCboxDimensions{shape=oval}
%\AMCcodeHspace=.5em
%\AMCcodeVspace=.5em
%\AMCcodeBoxSep=.1em
%
%\AMCinterBrep=0.5ex
\AMCinterBquest=1.pt
%
%\baremeDefautM{formula=NBC-NMC, p=0}
%\baremeDefautS{formula=NBC-NMC, p=0}
%\bareme{formula=NBC-NMC}
%\bareme{formula=(NB==N?NBC:NBC+NMC==N ? 0 : NBC-NMC),p=0,e=0,v=0}
%
%\AMCformHSpace=.1em
%\AMCformVSpace=0.2ex
%
\element{Attention}{ \hspace{1.5em} \tmtextbf{ \large(N.B. : Répondez à la question \underline{sans}  cocher les cases)}}


\AMCrandomseed{1527384}
%-----------------------------------------------------

% -----------------------------------------------
\element{groupA}{
\begin{question}{Q001}
Le Maroc est un payé Afrecain
\begin{choicescustom} \bareme{formula=NBC-NMC}
\bonne{Vrai }
\mauvaise{faux}
\end{choicescustom}
\end{question}
}
\element{groupA}{
\begin{question}{Q002}
Question 2
\begin{choicescustom} \bareme{formula=NBC-NMC}
\bonne{Vrai }
\mauvaise{Faux}
\end{choicescustom}
\end{question}
}
\element{groupA}{
\begin{question}{Q003}
Question 3
\begin{choicescustom} \bareme{formula=NBC-NMC}
\mauvaise{Faux}
\bonne{Vrai }
\end{choicescustom}
\end{question}
}
\element{groupA}{
\begin{question}{Q004}
Question 4
\begin{choicescustom} \bareme{formula=NBC-NMC}
\bonne{Faux}
\mauvaise{Vrai }
\end{choicescustom}
\end{question}
}

%----------------------------------------------
\element{general2}{
  \begin{questionmult}{QRC1}    
Question multiple avec deux bonnes réponses.
    \begin{reponses}   \bareme{formula=(NB==N?NBC:NBC+NMC==N ? 0 : NBC-NMC),p=0,e=0,v=0}
	\bonne{Bonne réponse 1}
	\bonne{Bonne réponse 2}
	     \mauvaise{Mauvaise réponse}
	     \mauvaise{Mauvaise réponse}
    \end{reponses}
  \end{questionmult}\vspace{2ex}
} 
\element{general2}{
  \begin{questionmult}{QRC2}    
Question multiple avec trois bonnes réponses.
    \begin{reponses}   
	\bonne{Bonne réponse 1}
	\bonne{Bonne réponse 2}
	\bonne{Bonne réponse 3}
	     \mauvaise{Mauvaise réponse}
    \end{reponses}
  \end{questionmult}\vspace{2ex}
} 
\element{general2}{
  \begin{question}{QRC3}    
Question simple (Une seule bonne réponse)
    \begin{reponses}   \bareme{formula=NBC-NMC}
	\bonne{Bonne réponse} 
	      \mauvaise{Mauvaise réponse}
	       \mauvaise{Mauvaise réponse}
    \end{reponses}
  \end{question}\vspace{2ex}
} 

%
%------------------------------- Question ouverte-----
\element{QuestionOuv}{
\vspace{1em}
\begin{question}[ ]{Lang1} 
Poser la question ici 

  \restituegroupe{Attention} 
  \AMCOpen{ backgroundcol=white,lines=2,dots=True}
  { \mauvaise[1]{ }\scoring{0}\mauvaise[2]{ }\scoring{1}\bonne[3]{}\scoring{2}}

\vspace{-1.5em}			       
 \end{question}
 
}
%----------------------------------------------------------
%---------------------------------------------------------
\element{QuestOuv}{

\restituegroupe{QuestionOuv}

}
%----------------------------------------------------------
%----------------------------------------------
% -----------------------------------------------
%
\onecopy{1}{
%
%\noindent
\begin{center}
\includegraphics[scale=0.35]{LOGOFSM.jpg}\\
\large\bf Département de XXX  \\

\,{\bf Filière XX : Semestre XX \hfill Session Ordinaire \hfill (Automne 2017/2018)} \\
   
\centering\bf ~ ~ ~ ~ ~ ~ ~ Module : XXXXXXXXXX   \hfill  Durée : 35 minutes~ ~ ~ ~ ~ ~ ~ \\
\end{center}
%\begin{center} Durée : 35 minutes. \end{center}

%%%%%%%%%%%%%  Début Saisie information étudiant 

{\setlength{\parindent}{0pt}\hspace*{\fill}\AMCcode{CNE}{10}\hspace*{\fill}
\begin{minipage}[b]{9.5cm}
\hspace{-1ex} $\Longleftarrow $ Codez les \textbf{10 chiffres} de votre \textbf{Code National \underline{d'Etudiant} } (\tmtextbf{C.N.E.}) ou \tmtextbf{Massar} ci-contre de la \tmtextbf{gauche} vers la \tmtextbf{droite}. \textbf{Attention} à ne noircir \textbf{qu'UN chiffre par colonne}. Pour le \tmtextbf{Massar} remplacer \textbf{la première} lettre en cochant le \tmtextbf{0} de \textbf{la première colonne} et remplir obligatoirement le cadre ci-dessous : 

\vspace{1ex}
\hfill\champnom{\fbox{
\begin{minipage}{.95\linewidth}
\vspace*{3mm}
\tmtextbf{Nom :}
\vspace*{.5cm}\dotfill

\tmtextbf{Prénom :}
\vspace*{.5cm}\dotfill

\tmtextbf{C.N.E : ou Massar :}
\vspace*{.5cm}\dotfill

\tmtextbf{Local :}
\vspace*{.5cm}\dotfill 
\tmtextbf{N° examen :}
\dotfill

\end{minipage}
}}\hfill\vspace{-1ex}
\end{minipage}\hspace*{\fill}

}
%%%%%%%%%%%%%  Fin Saisie information étudiant 


%
%\vspace{2mm}
\begin{center}\em
\begin{center} \large \bf \emph{Consignes} \end{center}
\begin{center} 
%Les questions faisant apparaître le symbole \multiSymbole{} présentent plusieurs bonnes réponses. 
Vous devez   \textbf{ \underline{colorier}} les cases au \textbf{\underline{stylo noir}} ou au \textbf{\underline{stylo bleu}} pour répondre aux questions. 
%Des points négatifs seront affectés aux mauvaises réponses.   
En cas d'erreur, il faut simplement effacer au « \textbf{blanco} » mais ne pas redessiner la case. \includegraphics[width=0.5\textwidth]{Att.jpg}
\end{center}
\end{center}

%\begin{center}
%\begin{minipage}[c]{.2\linewidth}
%\includegraphics[width=0.5\textwidth]{Att.jpg}
%\end{minipage}
%\end{center}
\hrule\vspace{0.2ex}


%----------------------------------------------------- 

\begin{center}
      \hrule\vspace{2mm}
	  \bf\Large \textsf{PREMIER TYPE : VRAI OU FAUX}
      \vspace{1mm}\hrule
\end{center}
%\begin{multicols}{2} \columnseprule=1.0pt
\AMCnumero{1}
\begin{flushleft}\hrule\vspace{2mm}\bf 1. Maroc :\vspace{1mm}\hrule\end{flushleft}
\melangegroupe{groupA}
\restituegroupe{groupA} 
%\end{multicols} 
%-------------------------------------------------------
%----------------------------------------------------- 

\begin{center}
      \hrule\vspace{2mm}
	  \bf\Large \textsf{DEUXIÈME TYPE : RÉPONSE A COCHER}
      \vspace{1mm}\hrule
\end{center}
\begin{multicols}{2} \columnseprule=1.0pt
\AMCnumero{1}
\melangegroupe{general2}
\restituegroupe{general2} 
\end{multicols} 
%-------------------------------------------------------

\restituegroupe{QuestOuv}

}
\end{document}
