\documentclass[a3paper,landscape,10pt]{article}
\usepackage[utf8x]{inputenc}
\usepackage[T1]{fontenc}
\usepackage[francais,bloc]{automultiplechoice}
\usepackage{multicol}

\geometry{hmargin=2.5cm,headheight=2.5cm,headsep=.3cm,footskip=1cm,top=2.5cm,bottom=2.5cm}
% les plus
\usepackage{tabto}   % ser à spécifier la distance de séparation
\usepackage{tikz}
\setlength\multicolsep{3pt}
\newcommand{\tmtextbf}[1]{{\bfseries{#1}}}
\usepackage{fp,numprint}
\usepackage{enumitem}



\begin{document}
\AMCrandomseed{1527384}

\def\AMCbeginQuestion#1#2{\par \hspace*{0mm}{\bf #1:} #2}
\def\AMCbeginAnswer#1#2{\tabto{13cm}\tabto{13cm}#2}

\AMCformHSpace=.5em    %.5em
\AMCinterBrep=5.5ex   % 5.5ex
\def\AMCanswer#1#2{#1\hspace{2mm}#2\hspace{1cm}}
%\def\AMCanswer#1#2{}
%%%%

\AMCinterIrep=2pt  %2
\AMCinterBrep=2pt  %2
\AMCinterIquest=2pt %2
\AMCinterBquest=1.5pt %1,5
%
%\baremeDefautM{formula=NBC-NMC, p=0}
%\bareme{b=1,m=0,e=0,v=0}


\element{Attention}{ \hspace{1.5em} (N.B. : Répondez à la question \textbf{sans  cocher les cases})}


\AMCrandomseed{1527384}

% -------------------- Compréhension -----------------
\element{general}{
  \begin{question}{Compr1}
\textbf{Dans ce texte il s'agit de :}
 %   \begin{multicols}{3}
    \begin{reponses}  \bareme{b=1,m=0,e=0,v=0}
	\bonne{La plantation des arbres pour gérer le  réchauffement climatique.}
		    \mauvaise{La déplantation des arbres en faveur du réchauffement climatique.}
			\mauvaise{La déplantation des arbres contre le réchauffement climatique.}
			\mauvaise{La plantation des arbres en faveur du réchauffement climatique.}
    \end{reponses}
   % \end{multicols}
  \end{question}
  \vspace{2ex}

}
%----------------------------------------
\element{general}{
  \begin{question}{Compr2}
\textbf{L'étude des chercheurs portent entre autre à identifier :}
 %   \begin{multicols}{3}
    \begin{reponses} \bareme{b=1,m=0,e=0,v=0}
	\bonne{Les arbres les plus efficaces à lutter contre le réchauffement climatique.}
		    \mauvaise{Les arbres les plus efficaces à s'adapter au réchauffement climatique.}
			\mauvaise{Les arbres les plus inefficaces à s'adapter au réchauffement climatique.}
			\mauvaise{Les arbres les plus inefficaces à lutter contre le réchauffement climatique.}
    \end{reponses}
 %   \end{multicols}
  \end{question}
  \vspace{2ex}

}
%----------------------------------------
\element{general}{
  \begin{question}{Compr3}
\textbf{La stratégie adoptée par les chercheurs  pour lutter contre ce phénomène est : }
 %   \begin{multicols}{3}
    \begin{reponses}   \bareme{b=1,m=0,e=0,v=0}
	\bonne{Le recoupement adéquat entre sites, arbres et les problèmes de santé.}
		   \mauvaise{Le recoupement inadéquat entre sites, arbres et les problèmes de santé.}
			\mauvaise{Le recoupement adéquat uniquement entre sites et arbres.}
			\mauvaise{Le recoupement inadéquat uniquement entre sites et arbres.}
    \end{reponses}
 %   \end{multicols}
  \end{question}
  \vspace{2ex}

}

\element{general}{
  \begin{question}{Compr4}
\textbf{L'objectif de cette étude :}
 %   \begin{multicols}{3}
    \begin{reponses}   \bareme{b=1,m=0,e=0,v=0}
	\bonne{Améliorer la santé et l'environnement des citoyens.}
		   \mauvaise{Sensibiliser à la santé et à l'environnement des citoyens.}
			\mauvaise{Nuire à la santé et à l'environnement des citoyens.}
			\mauvaise{Gérer la santé et l'environnement des citoyens.}
    \end{reponses}
 %   \end{multicols}
  \end{question}
  \vspace{2ex}

}
%----------------------------------------
\element{Compre}{
\restituegroupe{general}
\melangegroupe{general}
}

%----------------------------------------

%
%------------------------------- Production écrite -----
\element{Langue}{
\vspace{1em}
\begin{question}[ ]{Langue1}
\textbf{Relevez du texte une définition scientifique. (2 points).}
\\
  \restituegroupe{Attention}
  \AMCOpen{backgroundcol=white,lines=2,dots=True}
{\mauvaise[1]{ }\scoring{0}\mauvaise[2]{ }\scoring{1}\mauvaise[3]{ }\scoring{1.5}\bonne[4]{ }\scoring{2}}
\vspace{-1.5em}
 \end{question}

}
%-----------------------------------------------------
%------------------------------- Production écrite -----

\element{Langue}{
\vspace{1em}
\begin{question}[ ]{Langue2}
\textbf{De quel type est- elle? (1 point).}
\\
  \restituegroupe{Attention}
  \AMCOpen{backgroundcol=white,lines=1,dots=True}
{\mauvaise[1]{ }\scoring{0}\bonne[2]{ }\scoring{1}}
\vspace{-1.5em}
 \end{question}

}

\element{Langue}{
\vspace{1em}
\begin{question}[ ]{Langue3}
\textbf{Justifiez votre réponse. (2 points).}
\\
  \restituegroupe{Attention}
  \AMCOpen{backgroundcol=white,lines=2,dots=True}
 {\mauvaise[1]{ }\scoring{0}\mauvaise[2]{ }\scoring{1}\bonne[3]{ }\scoring{2}}
\vspace{-1.5em}
 \end{question}

}
%-----------------------------------------------------
%------------------------------- Production écrite -----
\element{Langue}{
\vspace{1em}
\begin{question}[ ]{Langue4}
{Reformulez votre définition. (2 points).}
\\
  \restituegroupe{Attention}
 %  \restituegroupe{Attention}
  \AMCOpen{backgroundcol=white,lines=2,dots=True}
{\mauvaise[1]{ }\scoring{0}\mauvaise[2]{ }\scoring{0.5}\mauvaise[3]{ }\scoring{1}\bonne[4]{ }\scoring{2}}
\vspace{-1.5em}
 \end{question}

}

\element{Langue}{
\vspace{1em}
\begin{question}[ ]{Langue5}
{Quel raisonnement est-il décliné dans le texte? (1 point).}
\\
  \restituegroupe{Attention}
  \AMCOpen{backgroundcol=white,lines=1,dots=True}
{\mauvaise[1]{ }\scoring{0}\bonne[2]{ }\scoring{1}}
\vspace{-1.5em}
 \end{question}

}


\element{Langue}{
\vspace{1em}
\begin{question}[ ]{Langue6}
{Quelle hypothèse de l'expérience est-elle étayée dans le texte? (2points).}
\\
  \restituegroupe{Attention}
  \AMCOpen{backgroundcol=white,lines=2,dots=True}
{\mauvaise[1]{ }\scoring{0}\mauvaise[2]{ }\scoring{1}\bonne[3]{ }\scoring{2}}
\vspace{-1.5em}
 \end{question}

}




\element{Langue}{
\vspace{1em}
\begin{question}[ ]{Langue7}
{Argumentez votre réponse par le biais de deux outils linguistiques.(2points).}
\\
  \restituegroupe{Attention}
  \AMCOpen{backgroundcol=white,lines=2,dots=True}
{\mauvaise[1]{ }\scoring{0}\mauvaise[2]{ }\scoring{1}\bonne[3]{ }\scoring{2}}
\vspace{-1.5em}
 \end{question}

}



%\element{Langue}{
%\vspace{1em}
%\begin{question}[ ]{Langue7}
%\textbf{Argumentez votre réponse par le biais d'un outil linguistique exprimant la généralité. (1 point).}
%\\
%  \restituegroupe{Attention}
%  \AMCOpen{backgroundcol=white,lines=1,dots=True}
%{\mauvaise[1]{ }\scoring{0}\bonne[2]{ }\scoring{1}}
%\vspace{-1.5em}
% \end{question}
%
%}

%\element{Langue}{
%\vspace{1em}
%\begin{question}[ ]{Langue8}
%\textbf{Dégagez du texte un troisième raisonnement scientifique.(1 point).}
%\\
%  \restituegroupe{Attention}
%  \AMCOpen{backgroundcol=white,lines=1,dots=True}
%{\mauvaise[1]{ }\scoring{0}\bonne[2]{ }\scoring{1}}
%\vspace{-1.5em}
% \end{question}
%
%}

\element{PartieLangue}{

\restituegroupe{Langue}

}

%-----------------------------------------------------
%------------------------------- Production écrite -----

%-------------------------------
%------------------------------- Production écrite -----

%-------------------------------

%----------------------------------------------------------



%------------------------------- Production écrite -----
\element{ProductionEcr}{
\vspace{1em}
\begin{question}[ ]{Production}
\textbf{Donnez la définition scientifique de l'écologie en respectant les critères de la vérifiabilité de la définition.}\\

  \restituegroupe{Attention}
  \AMCOpen{backgroundcol=white,lines=4,dots=True}
  {\mauvaise[1]{ }\scoring{0}\mauvaise[2]{ }\scoring{1}\mauvaise[3]{}\scoring{2}\mauvaise[4]{ }\scoring{3}\bonne[5]{ }\scoring{4}}
\vspace{-1.5em}
 \end{question}

}
%---------------------------------------------------------
\element{Produc}{

\restituegroupe{ProductionEcr}

}
%----------------------------------------------------------

%%%%%%%%%%%%%%%%%%%%%%%%%%%%%%%%%%%%%%%%%%%%%%%%%%%%%%%%%%%%%%%%%%%%%%%%%%%%%%%%%%%%%%%%%%%%%%%%%%%%%%%%%%%%%%%%%%%%%%%%%%%%%%%


%%% fabrication des copies

\exemplaire{1}{

%%% debut de l'en-tête des copies : QCM
\begin{multicols}{2}

\columnseprule=1.0pt
\noindent
\begin{center}
\includegraphics[scale=0.5]{Logo.png} \\
%\Large\bf Département de langue et communication

\centering\large\bf Filières SVTU  Semestre I \\ Session Ordinaire (2021/2022) \\ Module : Langue et Terminologie \end{center}
\begin{center} Durée : 1 heure et demi. \end{center}

%%%%%%%%%%%%%  Début Saisie information étudiant

{
\setlength{\parindent}{0pt}\hspace*{\fill}\AMCcode{CNE}{10}\hspace*{\fill}
\begin{minipage}[b]{8.5cm}
\hspace{-1ex} $\Longleftarrow $ Codez les \textbf{10 chiffres} de votre \textbf{Code National \underline{d'Etudiant} } (\tmtextbf{C.N.E.}) ou \tmtextbf{Massar} ci-contre de la \tmtextbf{gauche} vers la \tmtextbf{droite}. \textbf{Attention} à ne noircir \textbf{qu'UN chiffre par colonne}. Pour le \tmtextbf{Massar} remplacer \textbf{la première} lettre en cochant le \tmtextbf{0} de \textbf{la première colonne} et remplir obligatoirement le cadre ci-dessous :

\vspace{1ex}
\hfill\champnom{\fbox{
\begin{minipage}{.9\linewidth}
\vspace*{5mm}
\tmtextbf{Nom :}
\vspace*{.5cm}\dotfill

\tmtextbf{Prénom :}
\vspace*{.5cm}\dotfill

\tmtextbf{C.N.E./Massar :}
\vspace*{.5cm}\dotfill

\tmtextbf{Local :}
\vspace*{.2cm}\dotfill
\tmtextbf{N° examen :}
\dotfill

\end{minipage}
}}\hfill\vspace{-1ex}\end{minipage}\hspace*{\fill}

}
%%%%%%%%%%%%%  Fin Saisie information étudiant
\vspace{-4mm}
\begin{center}\em
\begin{center} \large \bf Consignes \end{center}
\begin{center}
Cet examen est sous forme de QCM (Questions à Choix Multiples). Pour chaque question, il y a plusieurs propositions, veuillez \textbf{ \underline{colorier}} \textbf{\underline{une seule case}} qui correspond à la bonne réponse au \textbf{\underline{stylo noir}} ou au \textbf{\underline{stylo bleu}}.
  En cas d'erreur, il faut simplement effacer au «\textbf{\underline{ blanco}} » mais ne pas redessiner la case.
 \begin{minipage}[c]{.5\linewidth}
\includegraphics[width=1\textwidth]{Att.jpg}
\end{minipage}
\begin{flushleft}
- Lire le texte plusieurs fois et attentivement.

- Les différentes parties de cet examen sont liées entre elles.

- \large{\textbf{Restez vigilants et bon courage !}}

%- Les réponses vous permettront de faire votre résumé.
  \end{flushleft}
  \end{center}

  Responsable : \textbf{Pr. Dalila BEGHDI }
\end{center}
%\vspace{0.5ex}
%----------------   Texte ---------------------------

\begin{flushleft}\hrule\vspace{4mm}\begin{center}
\large{ \textbf{Planter des arbres est envisagé depuis quelque temps comme un moyen efficace pour limiter les effets du réchauffement climatique. Encore faut-il savoir où et quels arbres planter? } }\\

\begin{flushleft}
\begin{flushleft}

~ ~ ~  Les  chercheurs de l’université Rice (États-Unis) se sont attachés à identifier les arbres qui  « fonctionneraient » probablement le mieux dans la ville, tenant compte de leur capacité à absorber le dioxyde de carbone (CO2) et d'autres polluants, de leur tendance à boire de l'eau, de leur aptitude à stabiliser le paysage pendant les inondations; ou encore à fournir une canopée: (la partie la plus élevée et qui abrite le plus de vie dans une forêt tropicale humide), pour atténuer la chaleur. Ils ont ensuite identifié les sites les plus appropriés à une plantation.  \\

~ ~ ~  Pour ce faire, ils ont compté sur des données recueillies au cours de la dernière décennie: des données de santé, de pollution, (etc). Puis ils ont fait appel à des analyses statistiques pour créer des cartes montrant les endroits où des plantations de masse auraient le plus fort impact,  et,  pour trouver les arbres les plus efficaces à absorber les polluants, à atténuer les inondations et à refroidir les îlots de chaleur urbains.
\\
~ ~ ~  C'est ainsi qu'ils ont mis au rebut* la plupart des 54 espèces indigènes. Pour n'en sélectionner que 17. Parmi eux, le chêne vert, extrêmement efficace à absorber les pollutions en tout genre. Le sycomore américain, quant à lui, excelle à contenir les inondations et à créer des zones d'ombre.  Ainsi que l'érable rouge: (arbre de la famille des acéracées et du genre Acer, aux fleurs et aux rameaux rouges, qui croît particulièrement bien dans les milieux humides.), et le chêne à feuilles de laurier. En croisant les informations relatives aux capacités de chaque essence d’arbre avec des données de santé liés à la pollution, les chercheurs montrent où il est préférable de planter des super-arbres en priorité.
\\
~ ~ ~  Se basant sur cette expérience, les chercheurs proposent aujourd'hui une stratégie en trois axes transposable à d'autres villes. Objectif : déterminer quels arbres sont les bons à planter, identifier les endroits où la plantation aura le plus d'impact sur la santé et sur l'environnement et s'engager avec la communauté pour faire du projet de plantation une réalité.
\\

\end{flushleft}


\begin{flushright}
      \underline{www.futura-sciences.com} , Décembre 2021. \end{flushright}

 ~ ~ ~\\
%\vspace*{0.2cm}

\textbf{GLOSSAIRE}\\
$^{*}$ Au rebut: se débarrasser de quelque chose qui est sans valeur ou inutilisable.\\
%$^{*}$ Unanimité: accord complet des opinions, des intentions.\\


\end{flushleft}
 \end{center}
%\vspace{0.2mm}
\end{flushleft}
%-----------------------------------------------------------
%-------------------------------------------------------
~ ~ ~\\

%\vspace*{0.5cm}

\begin{flushleft}\hrule\vspace{2mm}\begin{center}\large\bf QUESTIONS DE COMPRÉHENSION (4 POINTS) \end{center}
\vspace{1mm}\hrule \end{flushleft}

POUR CHAQUE QUESTION, COLORIEZ \textbf{\underline{UNE SEULE}} CASE. MERCI\\

\restituegroupe{Compre}
\melangegroupe{Compre}     % Pour ne pas mélanger
\AMCnumero{1}
%-------------------------------------------------------

%---------------------------------------------{\mauvaise[1]{ }\scoring{0}\bonne[2]{ }\scoring{1}}---------
%\pagebreak
%\melangegroupe{PartieA}           % Pour mélanger
%\restituegroupe{PartieA}

\AMCnumero{1}
%~ ~ ~\\
%\vspace*{0.5cm}
%\newpage
%-------------------------------------------------------

%-------------------------------------------------------


%-------------------------------------------------------
\begin{flushleft}\hrule\vspace{2mm}\begin{center}\large\bf QUESTIONS LANGUE ET COURS  (12 POINTS). \end{center}
\vspace{1mm}\hrule \end{flushleft}
%\textbf{Remplacez les connecteurs soulignés par leurs équivalents en apportant les modifications quand c'est nécessaire.}
\restituegroupe{PartieLangue}
\AMCnumero{1}


%------------------------------------------------------


%-------------------------------------------------------
\begin{flushleft}\hrule\vspace{2mm}\begin{center}\large\bf PRODUCTION ÉCRITE  (4  POINTS) \end{center}
\vspace{1mm}\hrule \end{flushleft}


\restituegroupe{Produc}
%-------------------------------------------------------






%\AMCcleardoublepage

\end{multicols}

}

\end{document}
