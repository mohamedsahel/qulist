\documentclass[a4paper,landscape,10pt]{article}
%\usepackage{xltxtra}
\usepackage[utf8x]{inputenc}
\usepackage[T1]{fontenc}
\usepackage[francais,bloc]{automultiplechoice}
\usepackage{multicol}
%\setmainfont{Linux Libertine O}
\geometry{hmargin=2.5cm,headheight=2.5cm,headsep=.3cm,footskip=1cm,top=2.5cm,bottom=2.5cm}
% les plus
\usepackage{tabto}
\usepackage{tikz}
\setlength\multicolsep{3pt}
ewcommand{\tmtextbf}[1]{{\bfseries{#1}}}
%
%%%
\begin{document}
\AMCrandomseed{1527384}
\AMCtext{none}{None of the above.}
\def\AMCbeginQuestion#1#2{\par \hspace*{0mm}{\bf #1:} #2}
%\def\AMCbeginAnswer#1#2{\tabto{13cm}\hspace{0mm}#2}
\def\AMCbeginAnswer#1#2{\tabto{13cm}\tabto{13cm}#2}
\AMCformHSpace=.5em
\AMCinterBrep=5.5ex
\def\AMCanswer#1#2{#1\hspace{2mm}#2\hspace{1cm}}
%%%%

\AMCinterIrep=2pt
\AMCinterBrep=2pt
\AMCinterIquest=2pt
\AMCinterBquest=1.5pt
%
%\baremeDefautM{formula=NBC-NMC, p=0}
%\baremeDefautS{e=0,v=0,b=1,m=-1}
%\bareme{b=2,m=0,e=0,v=0}
%\AMCformHSpace=.1em
%\AMCformVSpace=0.2ex
% \bareme{formula=(NB==N?NBC:NBC+NMC==N ? 0 : NBC/2-NMC/2),p=0,e=0,v=0}%

\AMCrandomseed{1527384}

% ---------------  Question  1  --------------- 
\element{general}{
\begin{question}{Q001}
 :
%\begin{vertical}{2}
    \begin{reponses} \bareme{formula=(NB==N?NBC:NBC+NMC==N ? 0 : NBC*1-NMC*0),p=0,e=0,v=0}
    
    \end{reponses}
%\end{vertical}
  \end{question}
 \vspace{2ex}

}
    
% ---------------  Question  2  --------------- 
\element{general}{
  \begin{question}{002}
 :
    \begin{reponses} \bareme{[object Object]}
      \wrong{Vrai}
      \correct{Faux}
    \end{reponses}
  \end{question}
 \vspace{2ex}
}


%----------------------------------------
\element{PartieA}{

\restituegroupe{general}
}


%%% fabrication des copies

\exemplaire{10}{

%%% debut de l'en-tête des copies : QCM
\begin{multicols}{3}

%\columnseprule=1.0pt
%oindent
\begin{center}
\includegraphics[scale=0.35]{LOGOFSM.jpg} \\

\vspace{2ex}

\noindent{\bf \emph{Département de Physics}  } \\

\vspace{2ex}

\noindent{\bf \emph{Filière : SMPC}  } \\

\vspace{2ex}

\noindent{\bf \emph{Session : Autumn 2022/2023}  } \\

\vspace{2ex}

\noindent{\bf \emph{Module : Module 1}  } \\

\vspace{2ex}

\noindent{\bf Durée : 1.5h } \\
\end{center}

%%%%%%%%%%%%%  Début Saisie information étudiant

{
\setlength{\parindent}{0pt}\hspace*{\fill}\AMCcode{CNE}{10}\hspace*{\fill}
\begin{center}


\begin{minipage}[b]{11cm}
$ \Longleftarrow $ Codez les \textbf{10 chiffres} de votre \textbf{Code National \underline{d'Etudiant} } (\tmtextbf{C.N.E.}) ou \tmtextbf{Massar} ci-contre de la \tmtextbf{gauche} vers la \tmtextbf{droite}. \textbf{Attention} à ne noircir \textbf{qu'UN chiffre par colonne}. Pour le \tmtextbf{Massar} remplacer \textbf{la première} lettre en cochant le \tmtextbf{0} de \textbf{la première colonne} et remplir obligatoirement le cadre ci-dessous :

\vspace{1ex}
\hfill\champnom{\fbox{
\begin{minipage}{.9\linewidth}
\vspace*{5mm}
\tmtextbf{Nom :}
\vspace*{.5cm}\dotfill

\tmtextbf{Prénom :}
\vspace*{.5cm}\dotfill

\tmtextbf{C.N.E. ou Massar :}
\vspace*{.5cm}\dotfill

\tmtextbf{Local :}
\vspace*{.2cm}\dotfill
\tmtextbf{N° examen :}
\dotfill

\end{minipage}
}}\hfill\vspace{-1ex}\end{minipage}\hspace*{\fill}
\end{center}
}
%%%%%%%%%%%%%  Fin Saisie information étudiant


%%% fin de l'en-tête QCM
\end{multicols}

\begin{center}\em
\begin{center} \large \bf \emph{Consignes} \end{center}
\begin{center}
%Les questions faisant apparaître le symbole \multiSymbole{} présentent plusieurs bonnes réponses.
Vous devez   \textbf{ \underline{colorier}} les cases au \textbf{\underline{stylo noir}} ou au \textbf{\underline{stylo bleu}} pour répondre aux questions.
%Des points négatifs seront affectés aux mauvaises réponses.
En cas d'erreur, il faut simplement effacer au « \textbf{blanco} » mais ne pas redessiner la case.
\end{center}
\end{center}

\begin{center}
\begin{minipage}[c]{.2\linewidth}
\includegraphics[width=1\textwidth]{Att.jpg}
\end{minipage}
\end{center}

\hrule\vspace{0.2ex}

\begin{multicols}{3}
\columnseprule=1.0pt
          % Pour mélanger
\restituegroupe{PartieA}

%\AMCnumero{1}


%\AMCnumero{1}


%\AMCcleardoublepage

\end{multicols}

}

\end{document}
